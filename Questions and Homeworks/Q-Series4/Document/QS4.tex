\documentclass[a4paper, 12pt]{article}

\usepackage{dblfnote}
\usepackage[perpage]{footmisc}
\usepackage{indentfirst}
\usepackage{framed}
\usepackage{tikz}
\usepackage{listings}[language=Python]
\usepackage{float}
%\usepackage{multicol}

\usepackage{setspace}
%\usepackage[skip=2mm, indent=17pt]{parskip}
\onehalfspacing
%\doublespacing


% Custom colors
\usepackage{color}

%\setcounter{secnumdepth}{0}

\usepackage[top=3cm, bottom=3cm, left = 2cm, right = 2cm]{geometry} 
\geometry{a4paper} 
\usepackage{url}
\usepackage{graphicx} 
\usepackage{amsmath,amssymb}  
\usepackage[hidelinks]{hyperref}
\usepackage[labelformat=empty]{caption}
\usepackage{xepersian}
\settextfont{XB Yas.ttf}
\usepackage[utf8]{inputenc}

%\usepackage{xepersian}

\DeclareFixedFont{\ttb}{T1}{txtt}{bx}{n}{12} % for bold
\DeclareFixedFont{\ttm}{T1}{txtt}{m}{n}{12}  % for normal

\definecolor{deepblue}{rgb}{0,0,0.5}
\definecolor{deepred}{rgb}{0.6,0,0}
\definecolor{deepgreen}{rgb}{0,0.5,0}

\newcommand\pythonstyle{\lstset{
		language=Python,
		basicstyle=\ttm,
		morekeywords={self},              % Add keywords here
		keywordstyle=\ttb\color{deepblue},
		emph={MyClass,__init__},          % Custom highlighting
		emphstyle=\ttb\color{deepred},    % Custom highlighting style
		stringstyle=\color{deepgreen},
		frame=single,                         % Any extra options here
		showstringspaces=false
}}


% Python environment
\lstnewenvironment{python}[1][]
{
	\pythonstyle
	\lstset{#1}
}
{}

% Python for external files
\newcommand\pythonexternal[2][]{{
		\pythonstyle
		\lstinputlisting[#1]{#2}}}

% Python for inline
\newcommand\pythoninline[1]{{\pythonstyle\lstinline!#1!}}




\begin{document}	
\noindent
\begin{minipage}[c]{5cm}
	\baselineskip=.7cm
	\begin{flushright}
		درس : یادگیری ماشین 
		\\
		دانشجو :
		امیرمحمد خرازی
		\\
		شماره دانشجویی :
		40152521002 
		\\
		استاد درس :  
		\href{mrezghi.ir}{دکتر منصور رزقی آهق}
	\end{flushright}
\end{minipage}
\hfill
\begin{minipage}[c]{3cm}
	\begin{center}
		\href{modares.ac.ir}{
			\includegraphics[width=2cm]{logo.png}}
	\end{center}	
\end{minipage}
\\[1mm]
\hrule depth .5mm \relax
\begin{flushright}
	پرسش‌های کلاسی سری چهارم
	\hfill
	دانشکده علوم ریاضی ، گروه علوم کامپیوتر، گرایش داده‌کاوی
	\\
	\vspace{5mm}
	گیت‌هاب درس (
	\href{https://github.com/A-M-Kharazi/Machine-Learning-TMU.git}{لینک}
	)
	\hfill
	گیت‌هاب این پرسش (
	\href{https://github.com/A-M-Kharazi/Machine-Learning-TMU/tree/main/Questions/Q-Series4}{لینک}
	)
\end{flushright}

\hrule depth .5mm\relax

%\tableofcontents
%\newpage

\section*{گزارش و توضیحات : }

توضیحات کامل داخل کد‌ها آورده شده است و در اینجا نیازی به توضیحات بیشتر نیست.

در جایی از سوال که از ما خواسته است منظم‌ سازی با استفاده نرم ۲ را به کمک روش‌های ماتریسی حل کنیم داریم :

مسئله : 
\[
\min ||\Phi W - t || _2^2 + \lambda ||W||_2^2
\]
معادل است با :

\[
\min ||[\Phi, \sqrt{\lambda}I]^T W - [t, 0]^T ||_2^2
\]

من برای اینکه به صورت ماتریس ستونی ننویسم همه را یک 
\lr{transpose}
گرفتم.


 دلیل ؟‌
 
 می‌دانیم 
 $||X||^2_2 = X^TX$
 فرض کنید معادله بالا بصورت زیر است 
 $||A||_2^2$
 که در آن 
 $A = [\Phi, \sqrt{\lambda}I]^T W - [t, 0]^T$.
 
 
 حال جواب آن برابر خواهد بود با 
 $A^TA$
 که یعنی 
 $\left( [\Phi, \sqrt{\lambda}I]^T W - [t, 0]^T\right)^T\left( [\Phi, \sqrt{\lambda}I]^T W - [t, 0]^T\right)$
 
 \begin{align*}
 	&\Longrightarrow \left( [\Phi, \sqrt{\lambda}I]^T W - [t, 0]^T\right)^T\left( [\Phi, \sqrt{\lambda}I]^T W - [t, 0]^T\right)\\
 	&\Longrightarrow \left( [\Phi W - t , \sqrt{\lambda}W]^T\right)^T \left([\Phi W - t , \sqrt{\lambda}W]^T\right)\\
 	&\Longrightarrow \left([\Phi W - t , \sqrt{\lambda}W]\right)\left([\Phi W - t , \sqrt{\lambda}W]^T\right)\\
 	&\Longrightarrow \left([\Phi W - t][\Phi W - t]^T\right) + \sqrt{\lambda}W \sqrt{\lambda}W^T\\
 	&\Longrightarrow \text{
 	هر جا ترانهاده است، عملا نیست و هر جا نیست، عملا هست. 
 }\\
&\Longrightarrow\text{
به این دلیل اینطوری شد که از اول همه را بصورت ترانهاده گرفته بودم.	
}\\
&\Longleftrightarrow \text{
بخش اول تعریف نرم ۲ و رگرسیون است و بخش دوم تعریف نرم ۲ و منظم ساز است
}\\
&\Longrightarrow ||\Phi W - t || _2^2 + \sqrt{\lambda}\sqrt{\lambda}||W||_2^2 = ||\Phi W - t || _2^2 + \lambda||W||_2^2
 \end{align*}

لذا معادل هستند.


جواب این بهینه سازی بصورت 
$(\Phi^T\Phi + \lambda I )W = \Phi^T t$
است.


دلیل ؟ 
فرض کنید تابع هدف بصورت زیر است : 
$||AW - T ||_2^2$
که در آن 
$A = [\Phi,\sqrt{\lambda}I]^T$
و
$T = [t, 0]^T$
. جواب این بهینه سازی، با حل دستگاه زیر صورت می‌گیرد :
$(A^TA)W = A^TT$

داریم :
\begin{align*}
	&\Longrightarrow (A^TA)W = A^TT \\
	&\Longrightarrow \left([\Phi^T,\sqrt{\lambda}I] [\Phi,\sqrt{\lambda}I]^T\right) W = [\Phi^T,\sqrt{\lambda}I][t, 0]^T\\
	&\Longrightarrow \left(\Phi^T \Phi  + \lambda I\right)W = \left(\Phi^T t + 0\right)\\
	&\Longrightarrow \left(\Phi^T \Phi  + \lambda I\right)W =  \left(\Phi^T t\right)
\end{align*}

همچنین نتیجه‌ای که از کد‌ها گرفتیم این بود که با نرم ۲ خیلی به داده‌های پرت حساس می‌شویم و عملا 
\lr{overfit}
داریم ، یا خیلی روی داده‌های آموزش متمرکز می‌شویم که اندازه 
$W$
خیلی زیاد می‌شود و روی داده‌های آزمایش بد عمل می‌کنیم که باز هم 
\lr{overfit}
داریم. این شرایط برای زمانی بود که از رگرسیون معمولی استفاده می‌کردیم. 


اگر از رگرسیون با تابع 
\lr{loss}
نرم ۱ استفاده کنیم به نقاط پرت کمتر حساسیم و در اغلب موارد نتیجه بهتری می‌گیریم. 

با منظم‌سازی های نرم ۱ و نرم ۲ می‌توانیم اندازه 
$W$
را کنترل کنیم که 
\lr{overfit}
نکنیم. 

با نرم ۱ در منظم‌سازی، دیدیم که اغلب 
$w \in W $
مقدار نزدیک به صفر می‌گیرند ولی بعضی از آن‌ها مقدار قابل توجهی دارند. با این روش منظم سازی می‌توانیم پیچیدگی مدل را کاهش دهیم، ابعاد را کمتر کنیم و غیره . که این منظم سازی 
$W$
را 
\lr{sparse}
می‌کند.


با نرم ۲ در منظم‌سازی، دیدیم که اغلب 
$w \in W$
مقدار نزدیک به هم دارند. یعنی عملا بیشتر 
$w$
ها در یک رنج هستند. این منظم سازی 
$W$
را 
\lr{smooth}
می‌کند.


با روش 
\lr{ElasticNet}
تلاش کردیم هر دو این موارد را  منظم‌سازی ها را به اجرا برسانیم.



\subsection*{مشکلات}

از نمونه به مشکلاتی که برخوردیم  :


\begin{itemize}
	\item 
	عدم همگرایی منظم‌سازی نرم ۱ 
	\item
	عدم همگرایی منظم‌سازی 
	\lr{ElasticNet}
	و برازش تنها 
	$W_0$
	در آن.
\end{itemize}


\end{document}


